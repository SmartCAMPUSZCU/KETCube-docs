% ------------------------------------------------------------
% ------------------------------------------------------------
  
% Insert the common KETCube Presentation Defines

% ------------------------------------------------------------
% ------------------------------------------------------------

\pdfminorversion=4
\documentclass[12pt]{beamer}

%\usetheme{KETCube}
\usepackage{resources/beamer/beamerthemeKETCube}

\usepackage{graphicx}
\usepackage{thumbpdf}
\usepackage{wasysym}
\usepackage{ucs}
\usepackage{substr}
\usepackage{xstring}
\usepackage{expl3,l3regex}
\usepackage[english]{babel}
\usepackage[utf8x]{inputenc}
\usepackage{lmodern,textcomp}
\usepackage{pgf,pgfarrows,pgfnodes,pgfautomata,pgfheaps,pgfshade}
\usepackage{wrapfig}
\usepackage{verbatim}
\usepackage{tikz}
\usetikzlibrary{tikzmark,fit}
\usepackage{color, colortbl}
\usepackage{tabu}
\usepackage{multicol}
\usepackage{fancyvrb}

\pdfinfo
{
  /Title       (The KETCube Project Tutorial)
  /Creator     (LaTeX)
  /Author      (The SmartCAMPUS Team)
}


% extract presenattion number to create include path for images
\ExplSyntaxOn
% Save a copy of \jobname
\tl_set:NV \NUMBER \c_sys_jobname_str
\regex_replace_once:nnN { [A-Za-z]*_[A-Za-z]*_ } { } \NUMBER
\regex_replace_once:nnN { _[A-Za-z]* } { } \NUMBER
\ExplSyntaxOff

% declare the path(s) where your graphic files are
\graphicspath{{resources/images/}{resources/images/presentations/\NUMBER/}}
  
  
%% \title{KETCube -- the Prototyping and Educational Platform for IoT}
%% \subtitle{}
%% \author{The SmartCAMPUS Team, University of West Bohemia}
%% \authorcontact[The SmartCAMPUS Team]{The SmartCAMPUS Team\\ UWB, Czech Republic\\www.smartcampus.cz}
%% \institute{University of West Bohemia}
%% \instituteaddress{Pilsen, Czech Republic}
%% 
%% % Use for conference/course/event identification
%% %\event[Euromicro Conference on Digital System Design] {Prague, 29\textsuperscript{th}-- 31\textsuperscript{st} August 2018}
%% 
%% % this will occupy list slide above "Thank you"
%% \summary{
%% \begin{itemize}
%%   \item A new Prototyping and Educational Platform\\for IoT -- {\bf KETCube}:
%%   \begin{itemize}
%%     \item {\bf accelerates} the {\bf education and R\&D} processes
%%     \item {\bf uses industry-level documentation and tools}
%%     \item {\bf is the point of integration}: speed-up of device validation, in-field testing and deployment
%%   \end{itemize}
%% \end{itemize}
%% }


% Redefine `\rowcolor` to allow a beamer overlay specifier
% New syntax: \rewcolor<overlay>[color model]{color}[left overhang][right overhang]
\makeatletter
% Open `\noalign` and check for overlay specification:
\def\rowcolor{\noalign{\ifnum0=`}\fi\bmr@rowcolor}
\newcommand<>{\bmr@rowcolor}{%
    \alt#1%
        {\global\let\CT@do@color\CT@@do@color\@ifnextchar[\CT@rowa\CT@rowb}% Rest of original `\rowcolor`
        {\ifnum0=`{\fi}\@gooble@rowcolor}% End `\noalign` and gobble all arguments of `\rowcolor`.
}
% Gobble all normal arguments of `\rowcolor`:
\newcommand{\@gooble@rowcolor}[2][]{\@gooble@rowcolor@}
\newcommand{\@gooble@rowcolor@}[1][]{\@gooble@rowcolor@@}
\newcommand{\@gooble@rowcolor@@}[1][]{\ignorespaces}
\makeatother

% Thank-You page

\AtEndDocument{\newgeometry{top=0cm, left=0cm, right=0cm, bottom=0cm}\begin{frame}[plain]\usebeamertemplate{endpage}\end{frame}\restoregeometry}

% Automatic frame title/subtitle from section/subsection

\addtobeamertemplate{frametitle}{
   \let\insertframetitle\insertsectionhead}{}
\addtobeamertemplate{frametitle}{
   \let\insertframesubtitle\insertsubsectionhead}{}

\makeatletter
  \CheckCommand*\beamer@checkframetitle{\@ifnextchar\bgroup\beamer@inlineframetitle{}}
  \renewcommand*\beamer@checkframetitle{\global\let\beamer@frametitle\relax\@ifnextchar\bgroup\beamer@inlineframetitle{}}
\makeatother


\title{KETCube User Configuration}
\subtitle{}
\author{The SmartCAMPUS Team, University of West Bohemia}
\authorcontact[The SmartCAMPUS Team]{The SmartCAMPUS Team\\ UWB, Czech Republic\\www.smartcampus.cz}
\institute{University of West Bohemia}
\instituteaddress{Pilsen, Czech Republic}
% Use for document title and subtitle
\thisdochead[KETCube -- the Prototyping and Educational Platform for IoT]{KETCube Platform Release 0.2}

% this will occupy the last - "Thank you" - slide
\summary{
  KETCube Firmware \& Terminal:
  
  \begin{itemize}
    \item {\bf easy-to-use} and {\bf easy-to-extend} IoT platform
    \item {\bf modular design}
    \item {\bf persistent} (EEPROM) {\bf and running} (RAM) configuration
  \end{itemize}
}

% ------------------------------------------------------------
% ------------------------------------------------------------
  
% Insert the common KETCube Presentation Head

% ------------------------------------------------------------
% ------------------------------------------------------------

\begin{document}
  
\mode
<all>
\newgeometry{top=0cm, left=0cm, right=0cm, bottom=0cm}
\begin{frame}[plain]
  \titlepage
\end{frame}
\addtocounter{framenumber}{-1}
\restoregeometry

\setbeamercolor{structure}{fg=riceBlue}

\mode
<all>

% ToC
\usebeamertemplate{toc}

% ------------------------------------------------------------
% ------------------------------------------------------------
  
% BEGIN of the KETCube Presentation Content

% ------------------------------------------------------------
% ------------------------------------------------------------


% ------------------------------------------------------------
% ------------------------------------------------------------
  
% BEGIN of the KETCube Presentation Content

% ------------------------------------------------------------
% ------------------------------------------------------------

\bsection{KETCube Modular Design}

\subsection{Board Modular Design}

\begin{frame}%[allowframebreaks]
  \centering
  
    \begin{columns}
      \begin{column}{0.6\paperwidth}
         \centering
         \begin{itemize}
           \item Re-use of existing standard for sensor development/evaluation boards: {\bf mikroBUS\texttrademark~socket} support (click-boards)
           \item Board-stack design: custom {\bf KETCube boards can be stacked} almost {\bf infinite} (pass-thru connectors)
           \item Small footprint enabling development board in-field usage
         \end{itemize}
      \end{column}
      \begin{column}{0.4\paperwidth}
         \centering
         \includegraphics[width=0.35\paperwidth]{thunder_click.jpg}
         \vfill
         \includegraphics[width=0.2\paperwidth]{ketCube_pinout.pdf}
       \end{column}
    \end{columns}
  
\end{frame}

\subsection{Firmware Modular Design}

\begin{frame}%[allowframebreaks]
  \centering
      
    \begin{columns}
      \begin{column}{0.5\paperwidth}
         \centering
         \begin{itemize}
           \item Simple {\bf architecture reflecting} IoT node {\bf use-case}
           \item {\bf Easy to} re-{\bf use} software modules
           \item Enable/disable software module in compile- and run-time
         \end{itemize}
         \vfill
         \includegraphics[width=0.45\paperwidth]{fw_stack.pdf}
      \end{column}
      
      \begin{column}{0.4\paperwidth}
           \begin{itemize}
           \item[$\rightarrow$] connect extension boards to KETCube mainBoard
           \item[$\rightarrow$] activate modules controlling the extension boards
         \end{itemize}
       \end{column}
    \end{columns}
  
\end{frame}

\bsection{KETCube Terminal}

\defverbatim[colored]\terminalCommandStructure{
\begin{lstlisting}[basicstyle=\ttfamily\scriptsize,keywordstyle=\color{red}]
[root command] [module name] [variable name] [parameters ...]

Example:
  >> show LoRa devEUI
\end{lstlisting}
}

\defverbatim[colored]\terminalRootCommands{
\begin{lstlisting}[basicstyle=\ttfamily\scriptsize,keywordstyle=\color{red}]
  about     Print ABOUT information: Copyright, License, ...
  help      Print HELP
  disable   Disable KETCube module
  enable    Enable KETCube module
  list      List available KETCube modules
  reload    Reload KETCube
  show      Show LoRa, SigFox ... parameters
  showr     Show LoRa, SigFox ... RUNNING parameters 
  set       Set LoRa, SigFox ... parameters
  setr      Set LoRa, SigFox ... RUNNING parameters
\end{lstlisting}
}

\begin{frame}%[allowframebreaks]
  Command Tree Structure:

  \terminalCommandStructure
  
  KETCube Root Commands:
  
  \terminalRootCommands

\end{frame}

\bsection{KETCube Modules}

\defverbatim[colored]\moduleListCommand{
\begin{lstlisting}[basicstyle=\ttfamily\scriptsize,keywordstyle=\color{red}]
>> list
 E  I  LoRa        LoRa radio
 E  R  HDC1080     On-board RHT sensor based on TI HDC1080
 D  R  batMeas     On-chip battery voltage measurement
 D  R  ADC         Measure mVolts on PA4

 ...
  
Module State: E == Module Enabled; D == Module Disabled
Module severity: N = NONE, R = ERROR; I = INFO; D = DEBUG
\end{lstlisting}
}

\begin{frame}%[allowframebreaks]
  Display Available modules:
  
  \moduleListCommand
  
  \scriptsize
  List of modules contains following informations:
  \begin{itemize}
    \item enabled/disabled -- state of the module \dots E/D
    \item severity level -- specifies how many information display in terminal \dots N/R/I/D
    \item module name
    \item module description
  \end{itemize}
  
\end{frame}


\bsection{KETCube Remote Terminal}

\begin{frame}%[allowframebreaks]
    
  \begin{itemize}
    \item a console application (currently in alpha) running on Windows or GNU/Linux opperating systems
    \item allows to execute the native KETCube commands remotely -- over LoRaWAN
    \item some of the -- potentially dangerous -- commands are disabled for the remote execution
  \end{itemize}
  
\end{frame}


% ------------------------------------------------------------
% ------------------------------------------------------------
  
% END of the KETCube Presentation Content

% ------------------------------------------------------------
% ------------------------------------------------------------



% ------------------------------------------------------------
% ------------------------------------------------------------
  
% Insert the common KETCube Presentation Tail

% ------------------------------------------------------------
% ------------------------------------------------------------


  
% ------------------------------------------------------------
% ------------------------------------------------------------
  
% END of the KETCube Presentation Content

% ------------------------------------------------------------
% ------------------------------------------------------------

\end{document}

