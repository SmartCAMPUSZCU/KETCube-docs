% ------------------------------------------------------------
% ------------------------------------------------------------
  
% Insert the common KETCube Presentation Defines

% ------------------------------------------------------------
% ------------------------------------------------------------

\pdfminorversion=4
\documentclass[12pt]{beamer}

%\usetheme{KETCube}
\usepackage{resources/beamer/beamerthemeKETCube}

\usepackage{graphicx}
\usepackage{thumbpdf}
\usepackage{wasysym}
\usepackage{ucs}
\usepackage{substr}
\usepackage{xstring}
\usepackage{expl3,l3regex}
\usepackage[english]{babel}
\usepackage[utf8x]{inputenc}
\usepackage{lmodern,textcomp}
\usepackage{pgf,pgfarrows,pgfnodes,pgfautomata,pgfheaps,pgfshade}
\usepackage{wrapfig}
\usepackage{verbatim}
\usepackage{tikz}
\usetikzlibrary{tikzmark,fit}
\usepackage{color, colortbl}
\usepackage{tabu}
\usepackage{multicol}
\usepackage{fancyvrb}
\usepackage{listings}

\pdfinfo
{
  /Title       (The KETCube Project Tutorial)
  /Creator     (LaTeX)
  /Author      (The SmartCAMPUS Team)
}


% extract presenattion number to create include path for images
\ExplSyntaxOn
% Save a copy of \jobname
\tl_set:NV \NUMBER \c_sys_jobname_str
\regex_replace_once:nnN { [A-Za-z]*_[A-Za-z]*_ } { } \NUMBER
\regex_replace_once:nnN { _[A-Za-z]* } { } \NUMBER
\ExplSyntaxOff

% declare the path(s) where your graphic files are
\graphicspath{{resources/images/}{resources/images/presentations/\NUMBER/}}
  
  
%% \title{KETCube -- the Prototyping and Educational Platform for IoT}
%% \subtitle{}
%% \author{The SmartCAMPUS Team, University of West Bohemia}
%% \authorcontact[The SmartCAMPUS Team]{The SmartCAMPUS Team\\ UWB, Czech Republic\\www.smartcampus.cz}
%% \institute{University of West Bohemia}
%% \instituteaddress{Pilsen, Czech Republic}
%% 
%% % Use for conference/course/event identification
%% %\event[Euromicro Conference on Digital System Design] {Prague, 29\textsuperscript{th}-- 31\textsuperscript{st} August 2018}
%% 
%% % this will occupy list slide above "Thank you"
%% \summary{
%% \begin{itemize}
%%   \item A new Prototyping and Educational Platform\\for IoT -- {\bf KETCube}:
%%   \begin{itemize}
%%     \item {\bf accelerates} the {\bf education and R\&D} processes
%%     \item {\bf uses industry-level documentation and tools}
%%     \item {\bf is the point of integration}: speed-up of device validation, in-field testing and deployment
%%   \end{itemize}
%% \end{itemize}
%% }


% Redefine `\rowcolor` to allow a beamer overlay specifier
% New syntax: \rewcolor<overlay>[color model]{color}[left overhang][right overhang]
\makeatletter
% Open `\noalign` and check for overlay specification:
\def\rowcolor{\noalign{\ifnum0=`}\fi\bmr@rowcolor}
\newcommand<>{\bmr@rowcolor}{%
    \alt#1%
        {\global\let\CT@do@color\CT@@do@color\@ifnextchar[\CT@rowa\CT@rowb}% Rest of original `\rowcolor`
        {\ifnum0=`{\fi}\@gooble@rowcolor}% End `\noalign` and gobble all arguments of `\rowcolor`.
}
% Gobble all normal arguments of `\rowcolor`:
\newcommand{\@gooble@rowcolor}[2][]{\@gooble@rowcolor@}
\newcommand{\@gooble@rowcolor@}[1][]{\@gooble@rowcolor@@}
\newcommand{\@gooble@rowcolor@@}[1][]{\ignorespaces}
\makeatother

% Thank-You page

\AtEndDocument{\newgeometry{top=0cm, left=0cm, right=0cm, bottom=0cm}\begin{frame}[plain]\usebeamertemplate{endpage}\end{frame}\restoregeometry}

% Automatic frame title/subtitle from section/subsection

\addtobeamertemplate{frametitle}{
   \let\insertframetitle\insertsectionhead}{}
\addtobeamertemplate{frametitle}{
   \let\insertframesubtitle\insertsubsectionhead}{}

\makeatletter
  \CheckCommand*\beamer@checkframetitle{\@ifnextchar\bgroup\beamer@inlineframetitle{}}
  \renewcommand*\beamer@checkframetitle{\global\let\beamer@frametitle\relax\@ifnextchar\bgroup\beamer@inlineframetitle{}}
\makeatother


\title{Introduction to KETCube}
\subtitle{}
\author{The SmartCAMPUS Team, University of West Bohemia}
\authorcontact[The SmartCAMPUS Team]{The SmartCAMPUS Team\\ UWB, Czech Republic\\www.smartcampus.cz}
\institute{University of West Bohemia}
\instituteaddress{Pilsen, Czech Republic}

% Use for document title and subtitle
\thisdochead[KETCube -- the Prototyping and Educational Platform for IoT]{KETCube Platform Release 0.2}

% this will occupy the last - "Thank you" - slide
\summary{
  The Prototyping and Educational Platform for IoT:
  
  \begin{itemize}
    \item {\bf accelerates} the {\bf education and R\&D} processes
    \item {\bf uses industry-level documentation and tools}
    \item {\bf is the point of integration}: speed-up of device validation, in-field testing and deployment
  \end{itemize}
}

% ------------------------------------------------------------
% ------------------------------------------------------------
  
% Insert the common KETCube Presentation Head

% ------------------------------------------------------------
% ------------------------------------------------------------

\begin{document}
  
\mode
<all>
\newgeometry{top=0cm, left=0cm, right=0cm, bottom=0cm}
\begin{frame}[plain]
  \titlepage
\end{frame}
\addtocounter{framenumber}{-1}
\restoregeometry

\setbeamercolor{structure}{fg=riceBlue}

\mode
<all>

% ToC
\usebeamertemplate{toc}

% ------------------------------------------------------------
% ------------------------------------------------------------
  
% BEGIN of the KETCube Presentation Content

% ------------------------------------------------------------
% ------------------------------------------------------------


% ------------------------------------------------------------
% ------------------------------------------------------------
  
% BEGIN of the KETCube Presentation Content

% ------------------------------------------------------------
% ------------------------------------------------------------

\bsection{Motivation}

\begin{frame}%[allowframebreaks]
    \begin{columns}
      \begin{column}{0.6\paperwidth}
         
         \begin{itemize}
           \item Diversity of IoT world brings challenges into both R\&D and Educational areas
           \item Technical -- R\&D
           \begin{itemize}
             \item {\bf point of integration:} simple to (re-)use HW and SW modules
             \item deployment to heterogeneous environments
             \item {\bf speed-up:} prototyping, validation and test series deployment
           \end{itemize}
         \end{itemize}
      \end{column}
      \begin{column}{0.4\paperwidth}
         \centering
         \includegraphics[width=0.35\paperwidth]{ketCube_all_photo_fullQ.jpg}
         \vfill
       \end{column}
    \end{columns}
  
  \flushleft
  \begin{columns}
      \begin{column}{0.95\paperwidth}
        \begin{itemize}
          \item Educational:
           \begin{itemize}
             \item {\bf mid-complexity:} balance the simplicity and insight (e.g.: Arduino vs. custom board power by FreeRTOS)
             \item industry-standard dev. style and documentation
           \end{itemize}
          \end{itemize}
        \end{column}
      \begin{column}{0.05\paperwidth}
         \centering
         ~
       \end{column}
    \end{columns}
  
\end{frame}

\bsection{KETCube Project Overview}

\subsection{Highlightes}

\begin{frame}%[allowframebreaks]
  \begin{itemize}
    \item KETCube is an Open Hardware project developed by the SmartCAMPUS Team at the University of West Bohemia in Pilsen
    \item The KETCube-related resources and design files are released under BSD-like license
    \item KETCube project resides on Github in several repositories under the SmartCAMPUS project: \url{https://github.com/SmartCAMPUSZCU}
  \end{itemize}
  
\end{frame}

\subsection{Core Parts}

\begin{frame}%[allowframebreaks]
\centering
      \begin{columns}
      \begin{column}{0.6\paperwidth}
         
         \begin{itemize}
      \item Main Board -- schematics and manufacturing data
      \item KETCube Firmware (v0.2) -- including project definitions for multiple IDEs and Doxygen-generated API documentation
      \item KETCube Documentation -- {\it Datasheet}, {\it App Notes} and Presentations
         \end{itemize}
      \end{column}
      \begin{column}{0.4\paperwidth}
         \centering
         \includegraphics[width=0.35\paperwidth]{ketCube_all_photo_fullQ.jpg}
         \vfill
       \end{column}
    \end{columns}
  \vspace{-0.5cm}
  \flushleft
  \begin{columns}
      \begin{column}{0.95\paperwidth}
         \begin{itemize}
             \item KETCube Tools -- KETCube-related tasks support
             \item KETCube Box -- 3D models and sources for KETCube cube-box
         \end{itemize}
      \end{column}
      \begin{column}{0.05\paperwidth}
         \centering
         ~
       \end{column}
    \end{columns}
  
\end{frame}

\bsection{KETCube Hardware in Detail}

\subsection{Board Design as the Point-of-Integration}

\begin{frame}%[allowframebreaks]
  \centering
  
    \begin{columns}
      \begin{column}{0.6\paperwidth}
         \centering
         \begin{itemize}
           \item Re-use of existing standard for sensor development/evaluation boards: {\bf mikroBUS\texttrademark~socket} support (click-boards)
           \item Board-stack design: custom {\bf KETCube boards can be stacked} almost {\bf infinite} (pass-thru connectors)
           \item Small footprint enabling development board in-field usage
         \end{itemize}
      \end{column}
      \begin{column}{0.4\paperwidth}
         \centering
         \includegraphics[width=0.35\paperwidth]{thunder_click.jpg}
         \vfill
         \includegraphics[width=0.2\paperwidth]{ketCube_pinout.pdf}
       \end{column}
    \end{columns}
  
\end{frame}

\subsection{Highlighted Parts}
\begin{frame}%[allowframebreaks]
  \centering
    
    \begin{columns}
      \begin{column}{0.6\paperwidth}
         \centering
         \begin{itemize}
           \item {\bf Murata Type ABZ}: STM32L0 MCU, SX1276 radio, manufacturer's support for LoRaWAN and Sigfox
           \item {\bf HDC1080}: Relative Humidity and Temperature sensor (RHT)
         \end{itemize}
      \end{column}
      \begin{column}{0.4\paperwidth}
         \centering
         \includegraphics[width=0.35\paperwidth]{thunder_click.jpg}
         \vfill
         \includegraphics[width=0.35\paperwidth]{typeabz.jpg}
       \end{column}
    \end{columns}
  
\end{frame}

\bsection{KETCube Firmware}

\subsection{Firmware as the Point-of-Integration}

\begin{frame}%[allowframebreaks]
  \centering
      
    \begin{columns}
      \begin{column}{0.6\paperwidth}
         \centering
         \begin{itemize}
           \item Simple {\bf architecture reflecting} IoT node {\bf use-case}
           \item {\bf Easy to} re-{\bf use} software modules
           \item Enable/disable software module in compile- and run-time
         \end{itemize}
         \vfill
         \includegraphics[width=0.5\paperwidth]{fw_stack.pdf}
      \end{column}
      \begin{column}{0.4\paperwidth}
           \includegraphics[width=0.3\paperwidth]{iotNodeStates.pdf}<1>
           \includegraphics[width=0.4\paperwidth]{firmwareStates.pdf}<2>
       \end{column}
    \end{columns}
  
\end{frame}

\subsection{Configuration}

\begin{frame}[fragile]%[allowframebreaks]
  \centering
    
\begin{itemize}
  \item Easy-to-use serial terminal interface:
  \begin{itemize}
    \item documented in {\it Datasheet}
    \item with built-in help
    \item includes command history
  \end{itemize}
\end{itemize}

\vspace{1cm}
    
  \begin{Verbatim}[frame=single, fontsize=\scriptsize]
> enable HDC1080 2
> enable LoRa
> set LoRa OTAA
> set LoRa appEUI 1122334455667788
> set LoRa appKey 11223344556677881122334455667788
> reload
 \end{Verbatim}
 
  
\end{frame}

\bsection{Surrounding Ecosystem}

\subsection{Industry-Standard Tools}

\begin{frame}%[allowframebreaks]

         \begin{itemize}
           \item Firmware: 
            \begin{itemize}
               \item no simplified custom solutions like Arduino IDE
               \item any STM32-ready compiler and Keil $\mu$Vision, Eclipse-based Atollic TrueSTUDIO or SW4STM32 or any Makefile-ready IDE (e.g. KDevelop)
               \item GNU Indent to enforce coding style
               \item Doxygen to generate annotation-based documentation
            \end{itemize}
           \item PCBs -- problems with tools compatibility
           \begin{itemize}
               \item currently schematics and manufacturing data are released
               \item planed: sample extension board projects in particular systems (Altium, KiCAD, Eagle, OrCAD) will be provided
            \end{itemize}
         \end{itemize}
  
\end{frame}

\subsection{Industry-Level Documentation}

\begin{frame}%[allowframebreaks]
      
\begin{columns}
      \begin{column}{0.6\paperwidth}
         \centering
         \begin{itemize}
           \item No README-only or Tutorial-only documentation
           \item Industry-inspired documentation style:
             \begin{itemize}
               \item platform {\bf Datasheet}
               \item {\bf Application notes}
               \item README -- used when advantageous or as a first step to create an Application Note
             \end{itemize}
           \item Doxygen to generate annotation-based documentation
         \end{itemize}
      \end{column}
      \begin{column}{0.4\paperwidth}
         \centering
           \includegraphics[width=0.4\paperwidth]{docs.pdf}\\
       \end{column}
    \end{columns}
  
\end{frame}


\bsection{Use Cases}

\subsection{Project Life Cycle}

\begin{frame}%[allowframebreaks]
  \centering
  
  \includegraphics[width=0.75\paperwidth]{dev_steps.pdf}
  
\end{frame}


% ToC
\usebeamertemplate{tocsection}

\subsection{Environmental Sensor LPWAN Node}
\begin{frame}%[allowframebreaks]
  \frametitle{Use Cases}
  \framesubtitle{Environmental Sensor LPWAN Node}
  \centering
        
\begin{columns}
      \begin{column}{0.5\paperwidth}
         \centering
         \begin{itemize}
           \item Out-of-the-box functionality
           \item No programming required
         \end{itemize}
      \end{column}
      \begin{column}{0.5\paperwidth}
         \centering
           \includegraphics[width=0.4\paperwidth]{rht.jpg}
       \end{column}
    \end{columns}
  
\end{frame}

\subsection{Presence Sensing Demonstrator}
\begin{frame}%[allowframebreaks]
  \centering
        
        
        
    \includegraphics[width=0.6\paperwidth]{cap_sensor.jpg}
  
   \begin{itemize}
           \item The KETCube extension board as a reduction to TI's FDC2214 dev-board
           \item Custom KETCube firmware module for FDC2214
   \end{itemize}
  
\end{frame}

% ------------------------------------------------------------
% ------------------------------------------------------------
  
% END of the KETCube Presentation Content

% ------------------------------------------------------------
% ------------------------------------------------------------



% ------------------------------------------------------------
% ------------------------------------------------------------
  
% Insert the common KETCube Presentation Tail

% ------------------------------------------------------------
% ------------------------------------------------------------


  
% ------------------------------------------------------------
% ------------------------------------------------------------
  
% END of the KETCube Presentation Content

% ------------------------------------------------------------
% ------------------------------------------------------------

\end{document}

