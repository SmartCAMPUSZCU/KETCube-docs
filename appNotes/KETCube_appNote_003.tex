% ------------------------------------------------------------
% ------------------------------------------------------------
  
% Insert the common KETCube AppNote Defines

% ------------------------------------------------------------
% ------------------------------------------------------------

\documentclass[twoside,a4paper]{refart}
\usepackage[utf8x]{inputenc}
%\usepackage[czech]{babel}
\usepackage[pdftex]{graphicx}
\usepackage{caption}% for \captionof
\usepackage{mwe}% contains example-image
% Definition of shortcuts for authornames
%  Sort alphabetically by surname

\newcommand{\JB}{Jan Bělohoubek}
\newcommand{\JC}{Jiří Čengery}
\newcommand{\JF}{Jaroslav Freisleben}
\newcommand{\PK}{Petr Kašpar}
\newcommand{\MU}{Matin Úbl}
\newcommand{\KV}{Kryštof Vaněk}
\newcommand{\JZ}{Jan Záruba}
\newcommand{\TP}{Tomáš Pokorný}

\usepackage[owncaptions]{vhistory}
\usepackage{hyperref}
%\usepackage[superscript,biblabel]{cite}
\usepackage{multirow}
\usepackage{wrapfig}
\usepackage{float}

\usepackage[table, x11names]{xcolor}
\usepackage{fancyvrb}
\usepackage{fontawesome}
\usepackage[many]{tcolorbox}
\usepackage{listings}
\usepackage{tikz}
\usepackage{verbatimbox}

\usepackage{array, booktabs, boldline} %
\usepackage{mathtools}

\pdfinfo
{
  /Title       (The KETCube Project AppNote)
  /Creator     (LaTeX)
  /Author      (The SmartCAMPUS Team)
}

% declare the path(s) where your graphic files are
% extract presenattion number to create include path for images
\ExplSyntaxOn
% Save a copy of \jobname
\tl_set:NV \NUMBER \c_sys_jobname_str
\regex_replace_once:nnN { [A-Za-z]*_[A-Za-z]*_ } { } \NUMBER
\ExplSyntaxOff

% declare the path(s) where your graphic files are
%\graphicspath{{resources/images/}{resources/appNotes/001/images/}}
\graphicspath{{resources/images/}{resources/appNotes/\NUMBER/images/}}
\graphicspath{{resources/images/}{resources/appNotes/003/images/}}

\newsavebox{\tempbox}
\newlength{\tempheight}

\newcommand\ToDo[1]{\textcolor{red}{ToDo: #1}}

\newcommand\docNote[1]{
\subsubsection*{~\hspace{0.2cm}\textcolor{cyan}{\Huge \faCommentingO\,}}
\vspace{-1.5cm}
\begin{tcolorbox}[breakable,colback=white,colframe=cyan,width=\dimexpr\textwidth+32mm\relax,enlarge left by=-32mm, title = Note]
\it \small #1
\end{tcolorbox}
}

\newcommand\docWarn[1]{
\subsubsection*{~\hspace{0.2cm}\textcolor{orange}{\Huge \faWarning\,}}
\vspace{-1.5cm}
\begin{tcolorbox}[breakable,colback=white,colframe=orange,width=\dimexpr\textwidth+32mm\relax,enlarge left by=-32mm, title = Warning]
\it #1
\end{tcolorbox}
}

\newcommand\docExample[1]{
\subsubsection*{~\hspace{0.2cm}\textcolor{gray}{\Huge \faCogs\,}}
\vspace{-1.5cm}
\begin{tcolorbox}[breakable,colback=white,colframe=gray,width=\dimexpr\textwidth+32mm\relax,enlarge left by=-32mm, title = Example]
#1
\end{tcolorbox}
}


\newenvironment{docCodeExample}
{
\subsubsection*{~\hspace{0.2cm}\textcolor{gray}{\Huge \faCogs\,}}
\vspace{-1.5cm}
\begin{tcolorbox}[breakable,colback=white,colframe=gray,width=\dimexpr\textwidth+32mm\relax,enlarge left by=-32mm, title = Example]
}
{
\end{tcolorbox}
}

\newenvironment{docCodeListing}
{
\subsubsection*{~\hspace{0.2cm}\textcolor{cyan}{\Huge \faStickyNoteO\,}}
\vspace{-1.5cm}
\begin{tcolorbox}[breakable,colback=white,colframe=cyan,width=\dimexpr\textwidth+32mm\relax,enlarge left by=-32mm, title = Listing]
}
{
\end{tcolorbox}
}

\newcommand\docFileName[1]{{\tt #1}}
\newcommand\docVarName[1]{{\tt #1}}

%skryt barevny obdelnik kolem odkazu
\hypersetup{
    colorlinks=false,
    pdfborder={0 0 0},
}

% vhistory
\renewcommand{\vhhistoryname}{Revision History}
\renewcommand{\vhchangename}{Note}
\renewcommand{\vhversionname}{Revision}
\renewcommand{\vhdatename}{Date}
\renewcommand{\vhauthorname}{Author}
\renewcommand \vhAuthorColWidth{0.8\hsize}
\renewcommand \vhChangeColWidth{1.2\hsize}

\DeclareRobustCommand{\UWBLogo}{%
   \begin{wrapfigure}{l}{2.1cm}
    \vspace{-1.35cm}
    \includegraphics[width=2cm]{ZCU_logo.pdf}
   \end{wrapfigure}
}


\title{\UWBLogo KETCube AppNote 003:\\ Voltage Measurement up to 100V DC (\vhCurrentVersion)}

\author{Author: \vhListAllAuthorsLongWithAbbrev}
\date{Version \vhCurrentVersion\ from \vhCurrentDate}

% ------------------------------------------------------------
% ------------------------------------------------------------
  
% Insert the common KETCube AppNote Head

% ------------------------------------------------------------
% ------------------------------------------------------------

\begin{document}
\pagenumbering{roman} 

\titlepage
\maketitle

% ------------------------------------------------------------
% ------------------------------------------------------------
  
% BEGIN of the KETCube appNote Content

% ------------------------------------------------------------
% ------------------------------------------------------------


% ------------------------------------------------------------
% ------------------------------------------------------------
  
% BEGIN of the KETCube appNote Content

% ------------------------------------------------------------
% ------------------------------------------------------------

\section*{About this Document}
{\it KETCube} \cite{ZCU:KETCube:05-2018} is the prototyping and demo platform developed at the Department of Technologies and Measurement (KET), University of West Bohemia in Pilsen. 


This document describes a simple yet powerful extension board with only 4 discrete-components for Voltage measurement up to 100V DC.


\setcounter{tocdepth}{1}
\tableofcontents
\clearpage

\listoffigures
\listoftables
\begin{versionhistory}
  \vhEntry{03/2018}{03.03.2018}{JB}{Initial version}
  \vhEntry{05/2018}{07.05.2018}{JB|KV|MU}{Text review, minor fixes}
\end{versionhistory}
% history table ... do not number
\setcounter{table}{0}

\clearpage 
\pagenumbering{arabic} 
\pagestyle{headings} 

\clearpage
\section{Measurement Principle}
The measurement is based on a simple Voltage divider -- see Figure \ref{fig:mp:voltDiv}.

\marginlabel{\captionof{figure}{Voltage Divider}\label{fig:mp:voltDiv}}
\raisebox{-\height}{\includegraphics[width=0.3\paperwidth]{volt_divider.pdf}}

For potential $U$, the following equations holds:

\begin{equation}
U = U_1 + U_2
\end{equation}

When $U_2$ is known, the following equations holds for $U$:

\begin{equation}
U = U_2 + \frac{U_2 \cdot R_1}{R_2}
\end{equation}



For current $I$, the following equation holds:

\begin{equation}
I = \frac{U_1}{R_1} = \frac{U_2}{R_2} = \frac{U}{R_1 + R_2}
\end{equation}

% ============================
\clearpage
\section{Extension Board}
The extension board contains the voltage divider and additional components to enhance circuit protection -- see Figure \ref{fig:eb:sch}.

\subsection{Schematic}
\marginlabel{\captionof{figure}{Extension board schematic and interfacing with KETCube}\label{fig:eb:sch}}
\raisebox{-\height}{\includegraphics[width=0.4\paperwidth]{schematic.pdf}}

\begin{table*}[!ht]
    \begin{tabular}{| p{2cm} | p{2cm} | p{1.5cm} |}
        \hline
        \rowcolor{SeaGreen3!30!} {\bf Reference} & {\bf Value} & {\bf Unit} \\
        \hline
        \hline
        R1 & 1M & $\Omega$ \\
        \hline
        R2 & 24k & $\Omega$ \\
        \hline
        R3 & 1k & $\Omega$ \\
        \hline
        D1 & 1N4007 & -- \\
        \hline
    \end{tabular}
    \addcontentsline{lot}{table}{Extension board components}
    \label{tab:eb:values}
   \end{table*}

\subsection{KETCube Interface}
  The extension board can be connected to KETCube via mikroBUS or KETCube socket. 
  
  Connect $ADCin$ to KETCube $AN$ pin and ground to KETCube $GND$ pin.
  
  Make sure, that $Vref$ on KETCube main board is connected to supply voltage -- see KETCube specification \cite{ZCU:KETCube:05-2018}.
  
\clearpage
\subsection{Operation}
  The measured voltage should be connected to pads $A$ (a positive terminal) and $B$ (a negative -- common -- terminal) respectively. The polarity of the measured voltage is $A \rightarrow B$. 

  When the measured voltage source is connected correctly, a small current flows from $A$ to $B$ and the partial voltage at $ADCin$ is measured to determine $A \rightarrow B$ voltage.

  When the measured voltage source is connected incorrectly ($B \rightarrow A$), no current flows through the circuit and the partial voltage at $ADCin$ is equal to 0V.
  
  KETCube \docKCModName{ADC} uses 12-bit ADC, this results in measurement resolution of $\approx$ 0.025V.

\docNote{The measured voltage source is constantly loaded by current $I \leq 100 ~\mu$A}
  
\subsection{Protections}
\subsubsection*{Over-Current}
  The resistor $R_3$ protects the MCU PIN.

\begin{equation}
  U_{ADCin} \leq U_{VDD} + U_{protection~diode~forward~voltage}
\end{equation}
  
\subsubsection*{Wrong Polarity}

When the polarity of voltage source connected to pads is wrong, i.e. $B \rightarrow A$, the diode $D1$ ensures, that the potential at $ADCin$ will be equal to KETCube ground.

% ============================
\clearpage
\section{Absolute Maximum Ratings}
  \begin{table*}[!ht]
    \hspace*{-4cm}
    \begin{tabular}{| p{3.5cm} | p{1.5cm} | p{2cm} | p{2cm} | p{2cm} | p{1cm} |}
        \hline
        \rowcolor{SeaGreen3!30!} {\bf Parameter} & {\bf Symbol} & {\bf MIN} & {\bf TYP} & {\bf MAX} & {\bf UNIT} \\
        \hline
        \hline
        Resolution & $U_{res}$ & -- & 0.025 & -- & V\\
        \hline
        \hline
        Supply Voltage & VDD & 2.5\footnotemark & \multicolumn{2}{l|}{see KETCube spec. \cite{ZCU:KETCube:05-2018}} & V \\
        \hline 
        1N4007 Forward voltage & $U_{DFW}$ & -- & 0.6 & -- & V\\
        \hline 
        1N4007 DC blocking voltage & $U_{DCB}$ & -- & -- & 1000 & V\\
        \hline 
        $A \rightarrow B$ voltage & $U_{A \rightarrow B}$ & $U_{DFW} + U_{res}$ & -- & 100 & V\\
        \hline 
        $B \rightarrow A$ voltage & $U_{B \rightarrow A}$ & -- & -- & $U_{DCB}$ & V\\
        \hline 
        $A \rightarrow B$ current & $I$ & -- & -- & 100 & $\mu$A\\
        \hline
    \end{tabular}
    \addcontentsline{lot}{table}{Absolute Maximum Ratings}
    \label{tab:spec:AMR}
   \end{table*}
   \footnotetext{Minimum supply voltage for full range measurement: 0V -- 100V}

% ============================
\clearpage
\section{KETCube Settings}
  In KETCube terminal (see KETCube specification \cite{ZCU:KETCube:05-2018}), enable \docKCModName{ADC} and configure any module for data delivery:
  
\begin{docCodeExample}
\begin{verbatim}
>> enable ADC
>> enable LoRa 
>> set LoRa appEUI 001122 ...

  ...
  
\end{verbatim}
\end{docCodeExample}

% ============================

\section{Measured Voltage Computation}
  The KETCube ADC module measure the $U_{ADCin}$ voltage, but the interesting value is $U_{A \rightarrow B}$. Thus $U_{A \rightarrow B}$ must be computed from  $U_{ADCin}$ by using the following equation:
  
\[
   U_{A \rightarrow B} = 
   \begin{dcases}
     \leq U_{DFW}, & \text{if } U_{ADCin} = 0\\
     U_{DFW} + U_{ADCin} + \frac{U_{ADCin} \cdot 10^6}{24 \cdot 10^3}, & \text{otherwise}
   \end{dcases}
\]

\docNote{The extension board enables measurement for $U_{A \rightarrow B} \geq U_{DFW}$ only.}

\docNote{When this extension board is used in conjunction with {\it LoRa} module, apply the above equation to the received value on the application server.}

  
% ============================

\clearpage
\bibliographystyle{IEEEtran}
\bibliography{IEEEabrv,resources/sources}

% ============================

%
% Include this license into all KETCube-related documentation
%
%

\clearpage

~

\vfill

\section*{Important Notice}

\subsection*{~\hspace{0.2cm}{\Huge \faCopyright\,} Copyright}
\vspace{-0.65cm}
\copyright ~2018 University of West Bohemia in Pilsen\\
All rights reserved.

\subsection*{~\hspace{0.2cm}{\Huge \faCogs\,} Developed by}
\vspace{-0.65cm}
The SmartCAMPUS Team\\
Department of Technologies and Measurement\\
Faculty of Electrical Engineering\\
www.smartcampus.cz/en $\mid$ www.zcu.cz/en

\subsection*{~\hspace{0.2cm}{\Huge \faCopy\,} License\footnotemark}
\vspace{-0.65cm}
\footnotetext{{\it University of Illinois/NCSA Open Source License} (\url{https://opensource.org/licenses/NCSA}) -- like the {\it Modified BSD License}}

Permission is hereby granted, free of charge, to any person obtaining a copy of this software and associated documentation files (the “Software”), to deal with the Software without restriction, including without limitation the rights to use, copy, modify, merge, publish, distribute, sublicense, and/or sell copies of the Software, and to permit persons to whom the Software is furnished to do so, subject to the following conditions:

\begin{itemize}
    \item[--] Redistributions of source code must retain the above copyright notice, this list of conditions and the following disclaimers.
    \item[--] Redistributions in binary form must reproduce the above copyright notice, this list of conditions and the following disclaimers in the documentation and/or other materials provided with the distribution.
    \item[--] Neither the names of The SmartCAMPUS Team, Department of Technologies and Measurement and Faculty of Electrical Engineering University of West Bohemia in Pilsen, nor the names of its contributors may be used to endorse or promote products derived from this Software without specific prior written permission. 
\end{itemize}
THE SOFTWARE IS PROVIDED “AS IS”, WITHOUT WARRANTY OF ANY KIND, EXPRESS OR IMPLIED, INCLUDING BUT NOT LIMITED TO THE WARRANTIES OF MERCHANTABILITY, FITNESS FOR A PARTICULAR PURPOSE AND NONINFRINGEMENT. IN NO EVENT SHALL THE CONTRIBUTORS OR COPYRIGHT HOLDERS BE LIABLE FOR ANY CLAIM, DAMAGES OR OTHER LIABILITY, WHETHER IN AN ACTION OF CONTRACT, TORT OR OTHERWISE, ARISING FROM, OUT OF OR IN CONNECTION WITH THE SOFTWARE OR THE USE OR OTHER DEALINGS WITH THE SOFTWARE. 

\end{document}

