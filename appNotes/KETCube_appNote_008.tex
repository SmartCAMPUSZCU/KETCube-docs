% ------------------------------------------------------------
% ------------------------------------------------------------
  
% Insert the common KETCube AppNote Defines

% ------------------------------------------------------------
% ------------------------------------------------------------

\documentclass[twoside,a4paper]{refart}
\usepackage[utf8x]{inputenc}
%\usepackage[czech]{babel}
\usepackage[pdftex]{graphicx}
\usepackage{caption}% for \captionof
\usepackage{mwe}% contains example-image
% Definition of shortcuts for authornames
%  Sort alphabetically by surname

\newcommand{\JB}{Jan Bělohoubek}
\newcommand{\JC}{Jiří Čengery}
\newcommand{\JF}{Jaroslav Freisleben}
\newcommand{\PK}{Petr Kašpar}
\newcommand{\MU}{Matin Úbl}
\newcommand{\KV}{Kryštof Vaněk}
\newcommand{\JZ}{Jan Záruba}
\newcommand{\TP}{Tomáš Pokorný}

\usepackage[owncaptions]{vhistory}
\usepackage{hyperref}
%\usepackage[superscript,biblabel]{cite}
\usepackage{multirow}
\usepackage{wrapfig}
\usepackage{float}

\usepackage[table, x11names]{xcolor}
\usepackage{fancyvrb}
\usepackage{fontawesome}
\usepackage[many]{tcolorbox}
\usepackage{listings}
\usepackage{tikz}
\usepackage{verbatimbox}

\usepackage{array, booktabs, boldline} %
\usepackage{mathtools}

\pdfinfo
{
  /Title       (The KETCube Project AppNote)
  /Creator     (LaTeX)
  /Author      (The SmartCAMPUS Team)
}

% declare the path(s) where your graphic files are
% extract presenattion number to create include path for images
\ExplSyntaxOn
% Save a copy of \jobname
\tl_set:NV \NUMBER \c_sys_jobname_str
\regex_replace_once:nnN { [A-Za-z]*_[A-Za-z]*_ } { } \NUMBER
\ExplSyntaxOff

% declare the path(s) where your graphic files are
%\graphicspath{{resources/images/}{resources/appNotes/001/images/}}
\graphicspath{{resources/images/}{resources/appNotes/\NUMBER/images/}}
\graphicspath{{resources/images/}{resources/appNotes/003/images/}}

\newsavebox{\tempbox}
\newlength{\tempheight}

\newcommand\ToDo[1]{\textcolor{red}{ToDo: #1}}

\newcommand\docNote[1]{
\subsubsection*{~\hspace{0.2cm}\textcolor{cyan}{\Huge \faCommentingO\,}}
\vspace{-1.5cm}
\begin{tcolorbox}[breakable,colback=white,colframe=cyan,width=\dimexpr\textwidth+32mm\relax,enlarge left by=-32mm, title = Note]
\it \small #1
\end{tcolorbox}
}

\newcommand\docWarn[1]{
\subsubsection*{~\hspace{0.2cm}\textcolor{orange}{\Huge \faWarning\,}}
\vspace{-1.5cm}
\begin{tcolorbox}[breakable,colback=white,colframe=orange,width=\dimexpr\textwidth+32mm\relax,enlarge left by=-32mm, title = Warning]
\it #1
\end{tcolorbox}
}

\newcommand\docExample[1]{
\subsubsection*{~\hspace{0.2cm}\textcolor{gray}{\Huge \faCogs\,}}
\vspace{-1.5cm}
\begin{tcolorbox}[breakable,colback=white,colframe=gray,width=\dimexpr\textwidth+32mm\relax,enlarge left by=-32mm, title = Example]
#1
\end{tcolorbox}
}


\newenvironment{docCodeExample}
{
\subsubsection*{~\hspace{0.2cm}\textcolor{gray}{\Huge \faCogs\,}}
\vspace{-1.5cm}
\begin{tcolorbox}[breakable,colback=white,colframe=gray,width=\dimexpr\textwidth+32mm\relax,enlarge left by=-32mm, title = Example]
}
{
\end{tcolorbox}
}

\newenvironment{docCodeListing}
{
\subsubsection*{~\hspace{0.2cm}\textcolor{cyan}{\Huge \faStickyNoteO\,}}
\vspace{-1.5cm}
\begin{tcolorbox}[breakable,colback=white,colframe=cyan,width=\dimexpr\textwidth+32mm\relax,enlarge left by=-32mm, title = Listing]
}
{
\end{tcolorbox}
}

\newcommand\docFileName[1]{{\tt #1}}
\newcommand\docVarName[1]{{\tt #1}}

%skryt barevny obdelnik kolem odkazu
\hypersetup{
    colorlinks=false,
    pdfborder={0 0 0},
}

% vhistory
\renewcommand{\vhhistoryname}{Revision History}
\renewcommand{\vhchangename}{Note}
\renewcommand{\vhversionname}{Revision}
\renewcommand{\vhdatename}{Date}
\renewcommand{\vhauthorname}{Author}
\renewcommand \vhAuthorColWidth{0.8\hsize}
\renewcommand \vhChangeColWidth{1.2\hsize}

\DeclareRobustCommand{\UWBLogo}{%
   \begin{wrapfigure}{l}{2.1cm}
    \vspace{-1.35cm}
    \includegraphics[width=2cm]{ZCU_logo.pdf}
   \end{wrapfigure}
}


\title{\UWBLogo KETCube AppNote 008:\\ KETCube Bootloader (\vhCurrentVersion)}

\author{Author: \vhListAllAuthorsLongWithAbbrev}
\date{Version \vhCurrentVersion\ from \vhCurrentDate}

% ------------------------------------------------------------
% ------------------------------------------------------------
  
% Insert the common KETCube AppNote Head

% ------------------------------------------------------------
% ------------------------------------------------------------

\begin{document}
\pagenumbering{roman} 

\titlepage
\maketitle

% ------------------------------------------------------------
% ------------------------------------------------------------
  
% BEGIN of the KETCube appNote Content

% ------------------------------------------------------------
% ------------------------------------------------------------


% ------------------------------------------------------------
% ------------------------------------------------------------
  
% BEGIN of the KETCube appNote Content

% ------------------------------------------------------------
% ------------------------------------------------------------

\section*{KETCube Bootloader: the STM Bootloader Re-Invented}
{\it KETCube} \cite{ZCU:KETCube:05-2018} is the prototyping and demo platform developed at the Department of Technologies and Measurement (KET), University of West Bohemia in Pilsen. 


This document describes the KETCube Bootloader. The bootloader is in part compatible with the standard ROM STM32 bootloader \cite{STM32:AN3155}.

\setcounter{tocdepth}{2}
\tableofcontents
\clearpage

\listoffigures
\listoftables
\begin{versionhistory}
  \vhEntry{0.2.0*}{8.6.2021}{JB}{Initial version}
\end{versionhistory}
% history table ... do not number
\setcounter{table}{0}

\clearpage 
\pagenumbering{arabic} 
\pagestyle{headings} 

\clearpage
\section{Supported Interfaces}
\subsection{UART1 Settings}

\begin{itemize}
  \item Tx PIN: PA9
  \item Rx PIN: PA10
  \item Default Baud Rate: 57600 bps
  \item Data bits: 8
  \item Stop bits: 1
  \item Parity: Even
  \item HW Flow control: No
\end{itemize}

\clearpage
\section{Supported Devices}\label{sec:supportedDevices}

\docNote{This bootloader is designed to be generic, however it is currently tested only on STM32L0[7,8][1-3] devices.}

\subsection{Adding a New Device}

To add a new device, insert the device definition to the \docFileName{device.h} file. For the device specification details, see Table 149 in \cite{STM32:AN2606}.

Always use the macro conditional expression to add a new device, e.g. \docVarName{\#if defined(STM32L071xx)}.

\docNote{To select the device, set \docVarName{DEVICE} variable in the Makefile to the device-speciffic string, e.g. to \docVarName{STM32L071xx} to select STM32l071 device.}

\clearpage
\section{Supported Flash Tools}

The primary tool intended for use in connection with this bootloader is {\it stm32flash\footnote{\url{https://sourceforge.net/projects/stm32flash/}}}.

\begin{docCodeExampleTitled}{Test Bootloader Connection}
\begin{verbatim}
$ ./stm32flash -c /dev/ttyUSB0 
stm32flash 0.5_KETCube

http://stm32flash.sourceforge.net/

Interface serial_posix: 57600 8E1
Version      : 0x01
Option 1     : 0x00
Option 2     : 0x00
Device ID    : 0x0447 (STM32L07xxx/08xxx)
- RAM        : Up to 20KiB  (8192b reserved by bootloader)
- Flash      : Up to 192KiB (size first sector: 32x128)
- Option RAM : 32b
- System RAM : 8KiB
\end{verbatim}
\end{docCodeExampleTitled}

\clearpage
\section{USART Bootloader Operation}

\marginlabel{\captionof{figure}{Bootloader operation sequence}\label{fig:bootOp:sequence}}
\raisebox{-\height}{\scalebox{0.7}{\begin{tikzpicture}[font=\small,thick]
 
\node[draw,
    minimum width=3.5cm,
    minimum height=1cm,
    align=center
] (block1) { 0x7F received on\\USARTx Rx pin };

\node[draw,
    below=of block1,
    minimum width=3cm,
    minimum height=1cm,
    align=center
] (block2) { USART1 selected };

\node[draw,
    below=of block2,
    minimum width=3cm,
    minimum height=3cm,
    align=center
] (block3) { Auto-baud rate sequence\\send ACK byte \& disable\\unused peripherals\footnotemark };

\node[coordinate,below=1cm of block3] (blockCmdWait) {};

\node[draw,
    diamond,
    below=1cm of blockCmdWait,
    minimum width=3cm,
    minimum height=2cm,
    align=center
] (block4) { Wait for a\\command };
 
\node[coordinate,below=1cm of block4] (blockCmdRecv) {};

\node[draw,
    below left=of blockCmdRecv,
    minimum width=3cm,
    minimum height=2cm,
    align=center
] (block6) { GET cmd\\routine };

\node[draw,
    right=of block6,
    minimum width=3cm,
    minimum height=2cm,
    align=center
] (block7) { READ cmd\\routine };

\node[draw,
    right=of block7,
    minimum width=3cm,
    minimum height=2cm,
    align=center
] (block8) { GO cmd\\routine };
 
\node[coordinate,below=1cm of block7] (blockCmdExec) {};
\node[coordinate,below=1cm of blockCmdExec] (blockCmdExec2) {};
\node[coordinate,left=8cm of blockCmdExec2] (blockCmdExec3) {};
 
% Arrows
\draw[-latex] (block1) edge (block2)
              (block2) edge (block3)
              (block3) -- (blockCmdWait)
              (blockCmdWait) -| (block4)
              (block4) edge (blockCmdRecv);
              
\draw[->] (blockCmdRecv) -| (block6);
\draw[->] (blockCmdRecv) -| (block7);
\draw[->] (blockCmdRecv) -| (block8);

\draw[] (block6) |- (blockCmdExec);
\draw[] (block7) |- (blockCmdExec);
\draw[] (block8) |- (blockCmdExec);

\draw[] (blockCmdExec) |- (blockCmdExec2)
        (blockCmdExec2) |- (blockCmdExec3);
\draw[->] (blockCmdExec3) |- (blockCmdWait);
 
\end{tikzpicture}}}
\footnotetext{autobaudrate is currently NOT supported by this bootloader}

\subsection{Auto-Baud Rate}
\docNote{The auto-baud rate is currently not implemented.}


\clearpage
\section{Boootloader Command Set}

  \begin{table*}[!ht]
    \hspace*{-4cm}
    \begin{tabular}{| p{4cm} | p{2cm} | p{7cm} |}
        \hline
        \rowcolor{SeaGreen3!30!} {\bf Command} & {\bf CMD Code} & {\bf Description} \\
        \hline
        \hline
        \nameref{cmd:get} & 0x00 & Get bootloader version and the list of commands supported by bootloader \\
        \hline
        \nameref{cmd:getVersion} & 0x01 & Get bootloader version \\
        \hline
        \nameref{cmd:getID} & 0x02 & Get the Chip ID \\
        \hline
        \nameref{cmd:readMem} & 0x11 & Read Memory \\
        \hline
        \nameref{cmd:writeMem} & 0x31 & Write Memory \\
        \hline
        \nameref{cmd:eraseMem} & 0x43 & Erase Memory \\
        \hline
    \end{tabular}
    \addcontentsline{lot}{table}{Bootloader Command Set}
    \label{tab:cmdset}
   \end{table*}

\clearpage
\subsection{Get} \label{cmd:get}

\marginlabel{\captionof{figure}{Get CMD Flow Diagram}\label{fig:cmd:get}}
\raisebox{-\height}{\scalebox{0.7}{\begin{tikzpicture}[font=\small,thick]
 
\node[draw,
    rounded rectangle,
    minimum width=3.5cm,
    minimum height=1cm,
    align=center
] (block1) { Start Get };

\node[draw,
    diamond,
    below=of block1,
    minimum width=3cm,
    minimum height=2cm,
    align=center
] (block2) { Received\\0x00,0xFF };

\node[draw,
    right=of block2,
    minimum width=3.5cm,
    minimum height=1cm,
    align=center
] (block3N) { Send NACK };

\node[draw,
    below=of block2,
    minimum width=3.5cm,
    minimum height=1cm,
    align=center
] (block3Y) { Send ACK };

\node[draw,
    below=of block3Y,
    minimum width=3.5cm,
    minimum height=1cm,
    align=center
] (block4) { Send the number of\\supported commands };

\node[draw,
    below=of block4,
    minimum width=3.5cm,
    minimum height=1cm,
    align=center
] (block5) { Send the\\bootloader version\footnotemark };

\node[draw,
    below=of block5,
    minimum width=3.5cm,
    minimum height=1cm,
    align=center
] (block6) { Send the\\supported commands };

\node[draw,
    below=of block6,
    minimum width=3.5cm,
    minimum height=1cm,
    align=center
] (block7) { Send ACK };
 
\node[draw,
    rounded rectangle,
    below=of block7,
    minimum width=3.5cm,
    minimum height=1cm,
    align=center
] (block8) { End of Get };

\draw[->] (block1) -- (block2);

\draw[->] (block2) -- (block3Y) node [pos=0.5,right,font=\footnotesize] { Yes };
\draw[->] (block2) -- (block3N) node [pos=0.3,above,font=\footnotesize] { No };

\draw[->] (block3Y) -- (block4);
\draw[->] (block4) -- (block5);
\draw[->] (block5) -- (block6);
\draw[->] (block6) -- (block7);
\draw[->] (block7) -- (block8);

\draw[->] (block3N) |- (block8);

\end{tikzpicture}}}
\footnotetext{The version is represented by a single byte, where MSB nibble represents the major version number, while the LSB nibble represents the minor version number. The version byte can be adjusted by setting the Makefile directive \docVarName{VERSION}}

\clearpage
\subsection{Get Version} \label{cmd:getVersion}

\marginlabel{\captionof{figure}{Get version CMD Flow Diagram}\label{fig:cmd:getVersion}}
\raisebox{-\height}{\scalebox{0.7}{\begin{tikzpicture}[font=\small,thick]
 
\node[draw,
    rounded rectangle,
    minimum width=3.5cm,
    minimum height=1cm,
    align=center
] (block1) { Start\\Get Version };

\node[draw,
    diamond,
    below=of block1,
    minimum width=3cm,
    minimum height=2cm,
    align=center
] (block2) { Received\\0x01,0xFE };

\node[draw,
    right=of block2,
    minimum width=3.5cm,
    minimum height=1cm,
    align=center
] (block3N) { Send NACK };

\node[draw,
    below=of block2,
    minimum width=3.5cm,
    minimum height=1cm,
    align=center
] (block3Y) { Send ACK };

\node[draw,
    below=of block3Y,
    minimum width=3.5cm,
    minimum height=1cm,
    align=center
] (block4) { Send the\\bootloader version\footnotemark };

\node[draw,
    below=of block4,
    minimum width=3.5cm,
    minimum height=1cm,
    align=center
] (block5) { Send 0x00\footnotemark };

\node[draw,
    below=of block5,
    minimum width=3.5cm,
    minimum height=1cm,
    align=center
] (block6) { Send 0x00\footnotemark[\value{footnote}] };

\node[draw,
    below=of block6,
    minimum width=3.5cm,
    minimum height=1cm,
    align=center
] (block7) { Send ACK };
 
\node[draw,
    rounded rectangle,
    below=of block7,
    minimum width=3.5cm,
    minimum height=1cm,
    align=center
] (block8) { End of\\Get Version };

\draw[->] (block1) -- (block2);

\draw[->] (block2) -- (block3Y) node [pos=0.5,right,font=\footnotesize] { Yes };
\draw[->] (block2) -- (block3N) node [pos=0.3,above,font=\footnotesize] { No };

\draw[->] (block3Y) -- (block4);
\draw[->] (block4) -- (block5);
\draw[->] (block5) -- (block6);
\draw[->] (block6) -- (block7);
\draw[->] (block7) -- (block8);

\draw[->] (block3N) |- (block8);

\end{tikzpicture}}}
\addtocounter{footnote}{-1}
\footnotetext{The version is represented by a single byte, where MSB nibble represents the major version number, while the LSB nibble represents the minor version number. The version byte can be adjusted by setting the Makefile directive \docVarName{VERSION}}
\addtocounter{footnote}{+1}
\footnotetext{STM bootloader protocol defines two Option Bytes, here the option bytes are fixed to 0x00}


\clearpage
\subsection{Get ID} \label{cmd:getID}

\marginlabel{\captionof{figure}{Get ID CMD Flow Diagram}\label{fig:cmd:getID}}
\raisebox{-\height}{\scalebox{0.7}{\begin{tikzpicture}[font=\small,thick]
 
\node[draw,
    rounded rectangle,
    minimum width=3.5cm,
    minimum height=1cm,
    align=center
] (block1) { Start\\Get ID };

\node[draw,
    diamond,
    below=of block1,
    minimum width=5cm,
    minimum height=3cm,
    align=center
] (block2) { Received\\0x02,0xFD\\\&\\Device Match\footnotemark };

\node[draw,
    right=of block2,
    minimum width=3.5cm,
    minimum height=1cm,
    align=center
] (block3N) { Send NACK };

\node[draw,
    below=of block2,
    minimum width=3.5cm,
    minimum height=1cm,
    align=center
] (block3Y) { Send ACK };

\node[draw,
    below=of block3Y,
    minimum width=3.5cm,
    minimum height=1cm,
    align=center
] (block4) { Send 0x01\footnotemark };

\node[draw,
    below=of block4,
    minimum width=3.5cm,
    minimum height=1cm,
    align=center
] (block5) { Send ID$_{MSB}$\footnotemark };

\node[draw,
    below=of block5,
    minimum width=3.5cm,
    minimum height=1cm,
    align=center
] (block6) { Send ID$_{LSB}$\footnotemark[\value{footnote}] };

\node[draw,
    below=of block6,
    minimum width=3.5cm,
    minimum height=1cm,
    align=center
] (block7) { Send ACK };
 
\node[draw,
    rounded rectangle,
    below=of block7,
    minimum width=3.5cm,
    minimum height=1cm,
    align=center
] (block8) { End of\\Get ID };

\draw[->] (block1) -- (block2);

\draw[->] (block2) -- (block3Y) node [pos=0.5,right,font=\footnotesize] { Yes };
\draw[->] (block2) -- (block3N) node [pos=0.3,above,font=\footnotesize] { No };

\draw[->] (block3Y) -- (block4);
\draw[->] (block4) -- (block5);
\draw[->] (block5) -- (block6);
\draw[->] (block6) -- (block7);
\draw[->] (block7) -- (block8);

\draw[->] (block3N) |- (block8);

\end{tikzpicture}}}
\addtocounter{footnote}{-2}
\footnotetext{The additional check is added to check if bootloader configuration match the device where it is executed}
\addtocounter{footnote}{+1}
\footnotetext{The first byte is fixed to 0x01 for STM32 devices}
\addtocounter{footnote}{+1}
\footnotetext{ID is sent MSB-first, the emulated device bootloader can be selected by the Makefile variable \docVarName{DEVICE} -- see Section \ref{sec:supportedDevices}}

\clearpage
\subsection{Read Memory} \label{cmd:readMem}

\marginlabel{\captionof{figure}{Read Memory CMD Flow Diagram}\label{fig:cmd:readMem}}
\raisebox{-\height}{\scalebox{0.7}{\begin{tikzpicture}[font=\small,thick]
 
\node[draw,
    rounded rectangle,
    minimum width=3.5cm,
    minimum height=1cm,
    align=center
] (block1) { Start\\Read Memory };

\node[draw,
    diamond,
    below=of block1,
    minimum width=3cm,
    minimum height=2cm,
    align=center
] (block2) { Received\\0x11,0xEE };

\node[draw,
    below=of block2,
    minimum width=3.5cm,
    minimum height=1cm,
    align=center
] (block3Y) { Send ACK\footnotemark };

\node[draw,
    below=of block3Y,
    minimum width=3.5cm,
    minimum height=2cm,
    align=center
] (block4) { Receive\\Start Address\\and Checksum\footnotemark };

\node[draw,
    diamond,
    below=of block4,
    minimum width=3cm,
    minimum height=2cm,
    align=center
] (block5) { Address\\\& Checksum };

\node[draw,
    below=of block5,
    minimum width=3.5cm,
    minimum height=1cm,
    align=center
] (block6) { Send ACK };

\node[draw,
    below=of block6,
    minimum width=3.5cm,
    minimum height=1cm,
    align=center
] (block7) { Receive\\Number of bytes\\and checksum\footnotemark };
 
\node[draw,
    diamond,
    below=of block7,
    minimum width=3cm,
    minimum height=2cm,
    align=center
] (block8) { Checksum };
 
\node[draw,
    below=of block8,
    minimum width=3.5cm,
    minimum height=1cm,
    align=center
] (block9) { Send ACK };

\node[draw,
    right=of block9,
    minimum width=3.5cm,
    minimum height=1cm,
    align=center
] (block9N) { Send NACK };

\node[draw,
    below=of block9,
    minimum width=3.5cm,
    minimum height=1cm,
    align=center
] (block10) { Send data to host };
 
\node[draw,
    rounded rectangle,
    below=of block10,
    minimum width=3.5cm,
    minimum height=1cm,
    align=center
] (block11) { End of\\Read Memory };

\draw[->] (block1) -- (block2);

\draw[->] (block2) -- (block3Y) node [pos=0.5,right,font=\footnotesize] { Yes };
\draw[->] (block2) -| (block9N) node [pos=0.3,above,font=\footnotesize] { No };

\draw[->] (block3Y) -- (block4);
\draw[->] (block4) -- (block5);
\draw[->] (block5) -- (block6) node [pos=0.5,right,font=\footnotesize] { Yes };
\draw[->] (block5) -| (block9N) node [pos=0.3,above,font=\footnotesize] { No };

\draw[->] (block6) -- (block7);
\draw[->] (block7) -- (block8);
\draw[->] (block8) -- (block9) node [pos=0.5,right,font=\footnotesize] { Yes };
\draw[->] (block8) -| (block9N) node [pos=0.3,above,font=\footnotesize] { No };

\draw[->] (block9) -- (block10);
\draw[->] (block10) -- (block11);

\draw[->] (block9N) |- (block11);

\end{tikzpicture}}}
\addtocounter{footnote}{-2}
\footnotetext{Read protection is not checked, ACK is always sent}
\addtocounter{footnote}{+1}
\footnotetext{4-byte address is received MSB-first, and the checksum is XOR of address bytes}
\addtocounter{footnote}{+1}
\footnotetext{Number of bytes - 1; the max value of this byte is 255 meaning 256 bytes will be sent; checksum is the complement to the number of bytes}

\clearpage
\subsection{Write Memory} \label{cmd:writeMem}

\marginlabel{\captionof{figure}{Write Memory CMD Flow Diagram}\label{fig:cmd:writeMem}}
\raisebox{-\height}{\scalebox{0.7}{\begin{tikzpicture}[font=\small,thick]
 
\node[draw,
    rounded rectangle,
    minimum width=3.5cm,
    minimum height=1cm,
    align=center
] (block1) { Start\\Write Memory };

\node[draw,
    diamond,
    below=of block1,
    minimum width=3cm,
    minimum height=2cm,
    align=center
] (block2) { Received\\0x31,0xCE };

\node[draw,
    below=of block2,
    minimum width=3.5cm,
    minimum height=1cm,
    align=center
] (block3Y) { Send ACK\footnotemark };

\node[draw,
    below=of block3Y,
    minimum width=3.5cm,
    minimum height=2cm,
    align=center
] (block4) { Receive\\Start Address\\and Checksum\footnotemark };

\node[draw,
    diamond,
    below=of block4,
    minimum width=3cm,
    minimum height=2cm,
    align=center
] (block5) { Address\\\& Checksum };

\node[draw,
    below=of block5,
    minimum width=3.5cm,
    minimum height=1cm,
    align=center
] (block6) { Send ACK };

\node[draw,
    below=of block6,
    minimum width=3.5cm,
    minimum height=1cm,
    align=center
] (block7) { Receive\\Number of bytes\\data and checksum\footnotemark };
 
\node[draw,
    diamond,
    below=of block7,
    minimum width=3cm,
    minimum height=2cm,
    align=center
] (block8) { Checksum\\\& address range\footnotemark };

 
\node[draw,
    below=of block8,
    minimum width=3.5cm,
    minimum height=1cm,
    align=center
] (block9) { Write data to\\destination memory };
 
\node[draw,
    below=of block9,
    minimum width=3.5cm,
    minimum height=1cm,
    align=center
] (block10) { Send ACK };

\node[draw,
    right=of block10,
    minimum width=3.5cm,
    minimum height=1cm,
    align=center
] (block9N) { Send NACK };
 
\node[draw,
    rounded rectangle,
    below=of block10,
    minimum width=3.5cm,
    minimum height=1cm,
    align=center
] (block11) { End of\\Write Memory };

\draw[->] (block1) -- (block2);

\draw[->] (block2) -- (block3Y) node [pos=0.5,right,font=\footnotesize] { Yes };
\draw[->] (block2) -| (block9N) node [pos=0.3,above,font=\footnotesize] { No };

\draw[->] (block3Y) -- (block4);
\draw[->] (block4) -- (block5);
\draw[->] (block5) -- (block6) node [pos=0.5,right,font=\footnotesize] { Yes };
\draw[->] (block5) -| (block9N) node [pos=0.3,above,font=\footnotesize] { No };

\draw[->] (block6) -- (block7);
\draw[->] (block7) -- (block8);
\draw[->] (block8) -- (block9) node [pos=0.5,right,font=\footnotesize] { Yes };
\draw[->] (block8) -| (block9N) node [pos=0.3,above,font=\footnotesize] { No };

\draw[->] (block9) -- (block10);
\draw[->] (block10) -- (block11);

\draw[->] (block9N) |- (block11);

\end{tikzpicture}}}
\addtocounter{footnote}{-3}
\footnotetext{Read protection is not checked, ACK is always sent}
\addtocounter{footnote}{+1}
\footnotetext{4-byte address is received MSB-first, and the checksum is XOR of address bytes}
\addtocounter{footnote}{+1}
\footnotetext{Number of bytes - 1; the max value of this byte is 255 meaning 256 bytes will be received, while the value must be multiple of four; checksum is computed as a XOR of the length and all received data bytes}
\addtocounter{footnote}{+1}
\footnotetext{Option byte writes are currently not supported}
  


\clearpage
\subsection{Erase Memory} \label{cmd:eraseMem}

\marginlabel{\captionof{figure}{Erase Memory CMD Flow Diagram}\label{fig:cmd:eraseMem}}
\raisebox{-\height}{\scalebox{0.7}{\begin{tikzpicture}[font=\small,thick]
 
\node[draw,
    rounded rectangle,
    minimum width=3.5cm,
    minimum height=1cm,
    align=center
] (block1) { Start\\Erase Memory };

\node[draw,
    diamond,
    below=of block1,
    minimum width=3cm,
    minimum height=2cm,
    align=center
] (block2) { Received\\0x43,0xBC };

\node[draw,
    below=of block2,
    minimum width=3.5cm,
    minimum height=1cm,
    align=center
] (block3Y) { Send ACK\footnotemark };

\node[draw,
    below=of block3Y,
    minimum width=3.5cm,
    minimum height=2cm,
    align=center
] (block4) { Receive\\the number of\\pages to be erased\footnotemark };

\node[draw,
    diamond,
    below=of block4,
    minimum width=3cm,
    minimum height=2cm,
    align=center
] (block5) { Received\\0xFF };

\node[draw,
    below=of block5,
    minimum width=3.5cm,
    minimum height=1cm,
    align=center
] (block6) { Receive\\page codes };

\node[draw,
    below=of block6,
    minimum width=3.5cm,
    minimum height=1cm,
    align=center
] (block7) { Receive\\the checksum };
 
\node[draw,
    diamond,
    below=of block7,
    minimum width=3cm,
    minimum height=2cm,
    align=center
] (block8) { Checksum };

 
\node[draw,
    below=of block8,
    minimum width=3.5cm,
    minimum height=1cm,
    align=center
] (block9) { Erase selected\\pages };
 
\node[draw,
    left=of block9,
    minimum width=3.5cm,
    minimum height=1cm,
    align=center
] (block6Y) { Global erase\\(Mass Erase) };
 
\node[draw,
    below=of block9,
    minimum width=3.5cm,
    minimum height=1cm,
    align=center
] (block10) { Send ACK };

\node[draw,
    right=of block10,
    minimum width=3.5cm,
    minimum height=1cm,
    align=center
] (block9N) { Send NACK };
 
\node[draw,
    rounded rectangle,
    below=of block10,
    minimum width=3.5cm,
    minimum height=1cm,
    align=center
] (block11) { End of\\Erase Memory };

\draw[->] (block1) -- (block2);

\draw[->] (block2) -- (block3Y) node [pos=0.5,right,font=\footnotesize] { Yes };
\draw[->] (block2) -| (block9N) node [pos=0.3,above,font=\footnotesize] { No };

\draw[->] (block3Y) -- (block4);
\draw[->] (block4) -- (block5);
\draw[->] (block5) -- (block6) node [pos=0.5,right,font=\footnotesize] { No };
\draw[->] (block5) -| (block6Y) node [pos=0.3,above,font=\footnotesize] { Yes };

\draw[->] (block6Y) |- (block10);

\draw[->] (block6) -- (block7);
\draw[->] (block7) -- (block8);
\draw[->] (block8) -- (block9) node [pos=0.5,right,font=\footnotesize] { Yes };
\draw[->] (block8) -| (block9N) node [pos=0.3,above,font=\footnotesize] { No };

\draw[->] (block9) -- (block10);
\draw[->] (block10) -- (block11);

\draw[->] (block9N) |- (block11);

\end{tikzpicture}}}
\addtocounter{footnote}{-1}
\footnotetext{Read protection is not checked, ACK is always sent}
\addtocounter{footnote}{+1}
\footnotetext{Number of pages to be erased decreased by 1; 0xFF is a special value and it starts Mass Erase}
  
  
% ------------------------------------------------------------
% ------------------------------------------------------------
  
% END of the KETCube appNote Content

% ------------------------------------------------------------
% ------------------------------------------------------------



% ------------------------------------------------------------
% ------------------------------------------------------------
  
% Insert the common KETCube appNote Tail

% ------------------------------------------------------------
% ------------------------------------------------------------

% ------------------------------------------------------------
% ------------------------------------------------------------
  
% END of the KETCube appNote Content

% ------------------------------------------------------------
% ------------------------------------------------------------


\clearpage
\bibliographystyle{IEEEtran}
\bibliography{IEEEabrv,resources/sources}

%
% Include this license into all KETCube-related documentation
%
%

\clearpage

~

\vfill

\section*{Important Notice}

\subsection*{~\hspace{0.2cm}{\Huge \faCopyright\,} Copyright}
\vspace{-0.65cm}
\copyright ~2018 University of West Bohemia in Pilsen\\
All rights reserved.

\subsection*{~\hspace{0.2cm}{\Huge \faCogs\,} Developed by}
\vspace{-0.65cm}
The SmartCAMPUS Team\\
Department of Technologies and Measurement\\
Faculty of Electrical Engineering\\
www.smartcampus.cz/en $\mid$ www.zcu.cz/en

\subsection*{~\hspace{0.2cm}{\Huge \faCopy\,} License\footnotemark}
\vspace{-0.65cm}
\footnotetext{{\it University of Illinois/NCSA Open Source License} (\url{https://opensource.org/licenses/NCSA}) -- like the {\it Modified BSD License}}

Permission is hereby granted, free of charge, to any person obtaining a copy of this software and associated documentation files (the “Software”), to deal with the Software without restriction, including without limitation the rights to use, copy, modify, merge, publish, distribute, sublicense, and/or sell copies of the Software, and to permit persons to whom the Software is furnished to do so, subject to the following conditions:

\begin{itemize}
    \item[--] Redistributions of source code must retain the above copyright notice, this list of conditions and the following disclaimers.
    \item[--] Redistributions in binary form must reproduce the above copyright notice, this list of conditions and the following disclaimers in the documentation and/or other materials provided with the distribution.
    \item[--] Neither the names of The SmartCAMPUS Team, Department of Technologies and Measurement and Faculty of Electrical Engineering University of West Bohemia in Pilsen, nor the names of its contributors may be used to endorse or promote products derived from this Software without specific prior written permission. 
\end{itemize}
THE SOFTWARE IS PROVIDED “AS IS”, WITHOUT WARRANTY OF ANY KIND, EXPRESS OR IMPLIED, INCLUDING BUT NOT LIMITED TO THE WARRANTIES OF MERCHANTABILITY, FITNESS FOR A PARTICULAR PURPOSE AND NONINFRINGEMENT. IN NO EVENT SHALL THE CONTRIBUTORS OR COPYRIGHT HOLDERS BE LIABLE FOR ANY CLAIM, DAMAGES OR OTHER LIABILITY, WHETHER IN AN ACTION OF CONTRACT, TORT OR OTHERWISE, ARISING FROM, OUT OF OR IN CONNECTION WITH THE SOFTWARE OR THE USE OR OTHER DEALINGS WITH THE SOFTWARE. 

\end{document}


\end{document}


