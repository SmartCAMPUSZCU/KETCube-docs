% ------------------------------------------------------------
% ------------------------------------------------------------
  
% Insert the common KETCube AppNote Defines

% ------------------------------------------------------------
% ------------------------------------------------------------

\documentclass[twoside,a4paper]{refart}
\usepackage[utf8x]{inputenc}
%\usepackage[czech]{babel}
\usepackage[pdftex]{graphicx}
\usepackage{caption}% for \captionof
\usepackage{mwe}% contains example-image
% Definition of shortcuts for authornames
%  Sort alphabetically by surname

\newcommand{\JB}{Jan Bělohoubek}
\newcommand{\JC}{Jiří Čengery}
\newcommand{\JF}{Jaroslav Freisleben}
\newcommand{\PK}{Petr Kašpar}
\newcommand{\MU}{Matin Úbl}
\newcommand{\KV}{Kryštof Vaněk}
\newcommand{\JZ}{Jan Záruba}
\newcommand{\TP}{Tomáš Pokorný}

\usepackage[owncaptions]{vhistory}
\usepackage{hyperref}
%\usepackage[superscript,biblabel]{cite}
\usepackage{multirow}
\usepackage{wrapfig}
\usepackage{float}

\usepackage[table, x11names]{xcolor}
\usepackage{fancyvrb}
\usepackage{fontawesome}
\usepackage[many]{tcolorbox}
\usepackage{listings}
\usepackage{tikz}
\usepackage{verbatimbox}

\usepackage{array, booktabs, boldline} %
\usepackage{mathtools}

\pdfinfo
{
  /Title       (The KETCube Project AppNote)
  /Creator     (LaTeX)
  /Author      (The SmartCAMPUS Team)
}

% declare the path(s) where your graphic files are
% extract presenattion number to create include path for images
\ExplSyntaxOn
% Save a copy of \jobname
\tl_set:NV \NUMBER \c_sys_jobname_str
\regex_replace_once:nnN { [A-Za-z]*_[A-Za-z]*_ } { } \NUMBER
\ExplSyntaxOff

% declare the path(s) where your graphic files are
%\graphicspath{{resources/images/}{resources/appNotes/001/images/}}
\graphicspath{{resources/images/}{resources/appNotes/\NUMBER/images/}}
\graphicspath{{resources/images/}{resources/appNotes/003/images/}}

\newsavebox{\tempbox}
\newlength{\tempheight}

\newcommand\ToDo[1]{\textcolor{red}{ToDo: #1}}

\newcommand\docNote[1]{
\subsubsection*{~\hspace{0.2cm}\textcolor{cyan}{\Huge \faCommentingO\,}}
\vspace{-1.5cm}
\begin{tcolorbox}[breakable,colback=white,colframe=cyan,width=\dimexpr\textwidth+32mm\relax,enlarge left by=-32mm, title = Note]
\it \small #1
\end{tcolorbox}
}

\newcommand\docWarn[1]{
\subsubsection*{~\hspace{0.2cm}\textcolor{orange}{\Huge \faWarning\,}}
\vspace{-1.5cm}
\begin{tcolorbox}[breakable,colback=white,colframe=orange,width=\dimexpr\textwidth+32mm\relax,enlarge left by=-32mm, title = Warning]
\it #1
\end{tcolorbox}
}

\newcommand\docExample[1]{
\subsubsection*{~\hspace{0.2cm}\textcolor{gray}{\Huge \faCogs\,}}
\vspace{-1.5cm}
\begin{tcolorbox}[breakable,colback=white,colframe=gray,width=\dimexpr\textwidth+32mm\relax,enlarge left by=-32mm, title = Example]
#1
\end{tcolorbox}
}


\newenvironment{docCodeExample}
{
\subsubsection*{~\hspace{0.2cm}\textcolor{gray}{\Huge \faCogs\,}}
\vspace{-1.5cm}
\begin{tcolorbox}[breakable,colback=white,colframe=gray,width=\dimexpr\textwidth+32mm\relax,enlarge left by=-32mm, title = Example]
}
{
\end{tcolorbox}
}

\newenvironment{docCodeListing}
{
\subsubsection*{~\hspace{0.2cm}\textcolor{cyan}{\Huge \faStickyNoteO\,}}
\vspace{-1.5cm}
\begin{tcolorbox}[breakable,colback=white,colframe=cyan,width=\dimexpr\textwidth+32mm\relax,enlarge left by=-32mm, title = Listing]
}
{
\end{tcolorbox}
}

\newcommand\docFileName[1]{{\tt #1}}
\newcommand\docVarName[1]{{\tt #1}}

%skryt barevny obdelnik kolem odkazu
\hypersetup{
    colorlinks=false,
    pdfborder={0 0 0},
}

% vhistory
\renewcommand{\vhhistoryname}{Revision History}
\renewcommand{\vhchangename}{Note}
\renewcommand{\vhversionname}{Revision}
\renewcommand{\vhdatename}{Date}
\renewcommand{\vhauthorname}{Author}
\renewcommand \vhAuthorColWidth{0.8\hsize}
\renewcommand \vhChangeColWidth{1.2\hsize}

\DeclareRobustCommand{\UWBLogo}{%
   \begin{wrapfigure}{l}{2.1cm}
    \vspace{-1.35cm}
    \includegraphics[width=2cm]{ZCU_logo.pdf}
   \end{wrapfigure}
}


\title{\UWBLogo KETCube AppNote 002:\\ LoRa Configuration (\vhCurrentVersion)}

\author{Author: \vhListAllAuthorsLongWithAbbrev}
\date{Version \vhCurrentVersion\ from \vhCurrentDate}

% ------------------------------------------------------------
% ------------------------------------------------------------
  
% Insert the common KETCube AppNote Head

% ------------------------------------------------------------
% ------------------------------------------------------------

\begin{document}
\pagenumbering{roman} 

\titlepage
\maketitle

% ------------------------------------------------------------
% ------------------------------------------------------------
  
% BEGIN of the KETCube appNote Content

% ------------------------------------------------------------
% ------------------------------------------------------------


% ------------------------------------------------------------
% ------------------------------------------------------------
  
% BEGIN of the KETCube appNote Content

% ------------------------------------------------------------
% ------------------------------------------------------------

\section*{About this Document}
{\it KETCube} \cite{ZCU:KETCube:05-2018} is the prototyping and demo platform developed at the Department of Technologies and Measurement (KET), University of West Bohemia in Pilsen. 


This document covers advanced configuration and testing of the KETCube LoRa module incorporating the LoRaWAN protocol Stack. 

The user-level documentation of the LoRa module is provided in AppNote 006 \cite{ZCU:KETCubeAppNote006:09-2019}.

\setcounter{tocdepth}{1}
\tableofcontents
\clearpage

\listoffigures
\listoftables
\begin{versionhistory}
  \vhEntry{03/2018}{03.03.2018}{JB}{Initial version}
  \vhEntry{05/2018}{07.05.2018}{JB|KV|MU}{Terminal configuration, text review, minor fixes}
  \vhEntry{0.2.0*}{xx.12.2019*}{JB}{KETCube-fw 0.2.0 update}
\end{versionhistory}
% history table ... do not number
\setcounter{table}{0}

\clearpage 
\pagenumbering{arabic} 
\pagestyle{headings} 

\clearpage
\clearpage
\section{Introduction}
  KETCube can be used as a LoRaWAN Class A, B or C device. The LoRaWAN stack supports 1.1.0 and 1.0.3 standards, whose can be selected in compile time, thus KETCube in it's binary form contains support for just a one of the standard versions.
  
  Many LoRaWAN parameters can be set dynamicaly -- through KETCube Terminal -- see KETCube specification \cite{ZCU:KETCube:05-2018} and AppNote 006 \cite{ZCU:KETCubeAppNote006:09-2019}. However, some of the core parameters can be set statically in compile-time.

  This document does not cover the LoRaWAN standard or terminology. Please refer the Lora Alliance websites\footnote{https://lora-alliance.org/}, LoRaWAN specification\footnote{https://lora-alliance.org/lorawan-for-developers} or see one of the knowledge bases on the web: the TTN documentation is a good starting point \footnote{https://www.thethingsnetwork.org/docs/lorawan/}.
  
\clearpage


\section{LoRaWAN Stack Compile-Time Configuration}

KETCube is designed as a memory-configured device. However, the LoRaWAN Stack can be (pre-)configured in compile-time. 

\subsection{LoRaWAN Stack Static Parameters}
To set-up some of the LoRaWAN stack parameters statically, use the \docFileName{Comissioning.h} file, which is the original Semtech's LoRaWAN configuration file. To enable static configuration from the \docFileName{Comissioning.h} file, set \docVarName{KETCUBE\_LORA\_SELCFG\_SELECTED} in \docFileName{ketCube\_lora.h} to\break \docVarName{KETCUBE\_LORA\_SELCFG\_STATIC}.

The folowing settings are available in \docFileName{Comissioning.h}:

\subsection*{Over the Air Activation}
The OTAA or ABP selection is configured here by setting the \break\docVarName{OVER\_THE\_AIR\_ACTIVATION} define to 1 for OTAA or to 0 for ABP.

\subsection*{Device EUI}
The \docVarName{LORAWAN\_DEVICE\_EUI} define represents a static devEUI if \break\docVarName{STATIC\_DEVICE\_EUI} macro is set to 1.

\subsection*{Application EUI and Application Key}
The define \docVarName{LORAWAN\_APPLICATION\_EUI} represents the Application EUI (AppEUI) for the OTAA mode.
The define \docVarName{LORAWAN\_APPLICATION\_KEY} represents the Application Key (AppKey) for the OTAA mode.

\subsection*{Device Address}
The define \docVarName{LORAWAN\_DEVICE\_ADDRESS} represents a static device address when \docVarName{STATIC\_DEVICE\_ADDRESS} define is set to 1.

\subsection*{LoRaWAN Session}
The define \docVarName{LORAWAN\_NWKSKEY} represents the LoRaWAN static Network Session Key (nwkSKey) used when ABP mode is used.
The define \docVarName{LORAWAN\_APPSKEY} represents the LoRaWAN static Application Session Key (appSKey) used when ABP mode is used.

\subsection{LoRaWAN Stack Static Behavioral Parameters}
Some of the important behavioral parameters should be configured in the \docFileName{ketCube\_lora.c} file. 

\subsection*{Adaptive Data Rate}
The define \docVarName{LORAWAN\_ADR\_ON} allows to enable/disable LoRaWAN adaptive data rate.

\subsection*{Acknowladgement Configuration}
Th define \docVarName{LORAWAN\_CONFIRMED\_MSG} allows to enable/disable LoRaWAN confirmed messages.

\subsection*{Application Ports} 
KETCube uses port numbers for pre-defined message types. The port numbers can be redefined in \docFileName{ketCube\_lora.c}.
\begin{itemize}
  \item \docVarName{LORAWAN\_APP\_PORT} -- for standard application uplink data
  \item \docVarName{LORAWAN\_ASYNCAPP\_PORT} -- for asynchronous application uplink data (e.g. push-buton)
  \item \docVarName{LORAWAN\_HEX\_DISPLAY\_PORT} -- for debugging; data received on this port are printed to terminal as HEX number(s)
  \item \docVarName{LORAWAN\_STRING\_DISPLAY\_PORT} -- for debugging; data received on this port are printed to terminal as string
  \item \docVarName{LORAWAN\_CUSTOM\_DATA\_PORT} -- data received on this port are processed by a callback function \docFnName{ketCube\_lora\_processCustomData()}
  \item \docVarName{LORAWAN\_REMOTE\_TERMINAL\_PORT} -- for remote terminal communication
\end{itemize}

\docNote{\docFnName{ketCube\_lora\_processCustomData()} function is defined \docVarName{weak} and can be redefined in user-code.}

\clearpage
\section{Application Testing}
\subsection{LoRaWAN Uplink Test}
In LoRaWAN, the term {\it uplink} denotes the direction of the data transmission from the LoRaWAN node to the LoRaWAN network server.

To test uplink, configure KETCube LoRaWAN parameters as described in sections above and use the same settings to configure the LoRaWAN node on your LoRaWAN network server.

Then enable some of the KETCube sensing modules and KETCube LoRa module and by using the KETCube Terminal interface. Don't forget to reload KETCube to apply configuration:

\begin{docCodeExampleTitled}{Basic Uplink Test Configuration}
\begin{verbatim}
>> enable BatMeas
>> enable HDC1080
>> enable LoRa 2
>> reload
\end{verbatim}
\end{docCodeExampleTitled}

After reload, you should observe incoming packets at your LoRaWAN server.

\subsection{LoRaWAN Downlink Test}
In LoRaWAN, the term {\it downlink} denotes the direction of the data transmission from the LoRaWAN network server to the LoRaWAN node.

For downlink tests, the following LoRaWAN ports are dedicated:\break \docVarName{LORAWAN\_HEX\_DISPLAY\_PORT} and \docVarName{LORAWAN\_STRING\_DISPLAY\_PORT} -- default values are 10 and 11. Data received on port 10 are displayed in KETCube Terminal as HEX string and data received on port 11 as string.

\begin{docCodeExampleTitled}{Basic Downlink Test Configuration}
\begin{verbatim}
>> enable LoRa 2
>> reload
\end{verbatim}
\end{docCodeExampleTitled}

After that, send data to KETCube LoRaWAN node to port 10 or 11 and watch KETCube Terminal.

\docNote{If you use the open {\it LoRa Server} ({https://www.loraserver.io/}), you can use the KETCube tool \docFileName{ketCube\_LoRaDownlink.py} located in the {\it KETCube-tools} repository. The script automates the process of sending debug downlink packets to KETCube's port 11.}

% ============================

\clearpage
\bibliographystyle{IEEEtran}
\bibliography{IEEEabrv,resources/sources}

% ============================

%
% Include this license into all KETCube-related documentation
%
%

\clearpage

~

\vfill

\section*{Important Notice}

\subsection*{~\hspace{0.2cm}{\Huge \faCopyright\,} Copyright}
\vspace{-0.65cm}
\copyright ~2018 University of West Bohemia in Pilsen\\
All rights reserved.

\subsection*{~\hspace{0.2cm}{\Huge \faCogs\,} Developed by}
\vspace{-0.65cm}
The SmartCAMPUS Team\\
Department of Technologies and Measurement\\
Faculty of Electrical Engineering\\
www.smartcampus.cz/en $\mid$ www.zcu.cz/en

\subsection*{~\hspace{0.2cm}{\Huge \faCopy\,} License\footnotemark}
\vspace{-0.65cm}
\footnotetext{{\it University of Illinois/NCSA Open Source License} (\url{https://opensource.org/licenses/NCSA}) -- like the {\it Modified BSD License}}

Permission is hereby granted, free of charge, to any person obtaining a copy of this software and associated documentation files (the “Software”), to deal with the Software without restriction, including without limitation the rights to use, copy, modify, merge, publish, distribute, sublicense, and/or sell copies of the Software, and to permit persons to whom the Software is furnished to do so, subject to the following conditions:

\begin{itemize}
    \item[--] Redistributions of source code must retain the above copyright notice, this list of conditions and the following disclaimers.
    \item[--] Redistributions in binary form must reproduce the above copyright notice, this list of conditions and the following disclaimers in the documentation and/or other materials provided with the distribution.
    \item[--] Neither the names of The SmartCAMPUS Team, Department of Technologies and Measurement and Faculty of Electrical Engineering University of West Bohemia in Pilsen, nor the names of its contributors may be used to endorse or promote products derived from this Software without specific prior written permission. 
\end{itemize}
THE SOFTWARE IS PROVIDED “AS IS”, WITHOUT WARRANTY OF ANY KIND, EXPRESS OR IMPLIED, INCLUDING BUT NOT LIMITED TO THE WARRANTIES OF MERCHANTABILITY, FITNESS FOR A PARTICULAR PURPOSE AND NONINFRINGEMENT. IN NO EVENT SHALL THE CONTRIBUTORS OR COPYRIGHT HOLDERS BE LIABLE FOR ANY CLAIM, DAMAGES OR OTHER LIABILITY, WHETHER IN AN ACTION OF CONTRACT, TORT OR OTHERWISE, ARISING FROM, OUT OF OR IN CONNECTION WITH THE SOFTWARE OR THE USE OR OTHER DEALINGS WITH THE SOFTWARE. 

\end{document}

